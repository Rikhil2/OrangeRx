\documentclass{article}
\usepackage{graphicx} % Required for inserting images

\title{oxygen model}
\author{Richard Beck}
\date{November 2025}

\begin{document}

\section*{Mathematical model}

\subsection*{State space and notation}

We model the population as a continuous-time multitype branching process with
cell states indexed by total chromosome number.
We distinguish two layers:
(i) \emph{pre–WGD} states with chromosome numbers $N_0 \in \mathcal{N}_0 = \{N_{\min},\dots,N_{\max}\}$,
and (ii) \emph{post–WGD} states with $N_1 \in \mathcal{N}_1 = \{N_{\min},\dots,N_{\max}\}$.
Let $x(t)$ be the vector of expected abundances across all states, ordered
so that
\[
x(t)
= 
\begin{pmatrix}
x_0(t) \\ x_1(t)
\end{pmatrix},
\qquad 
x_0(t) \in \mathbb{R}^{R_0},\;
x_1(t) \in \mathbb{R}^{R_1},
\]
where $R_0 = |\mathcal{N}_0|$ and $R_1 = |\mathcal{N}_1|$.
The dynamics follow the linear ODE
\[
\frac{d}{dt} x(t)
= 
G \, x(t),
\]
where $G$ is the infinitesimal generator of the branching process.

\subsection*{Growth-rate function}

Each chromosome number $N$ is mapped to a ``ploidy index''
\[
P = \max\!\left(1, \frac{N}{N_{\mathrm{unit}}}\right),
\]
where $N_{\mathrm{unit}}=22$ for diploid cells.
The division rate under oxygen level $O_2\in[0,1]$ is
\[
\lambda(N,O_2)
=
R \;
\frac{
1 - (1-O_2)\,
\displaystyle\frac{\beta(P-1)}{1+\beta(P-1)}
}{
1 + \eta (P-1)
}.
\]
The numerator captures an O$_2$-dependent penalty that increases with ploidy
and saturates as $P\to\infty$.
The denominator represents an O$_2$-independent burden (e.g.\ metabolic cost,
chromosomal instability).
We enforce $\lambda\ge 0$.

Let $\lambda_0(N_0)$ be growth rates on the pre-WGD layer and
$\lambda_1(N_1)$ the post-WGD rates, evaluated using the same functional form.

\subsection*{Missegregation distribution}

A mother cell at chromosome number $N$ missegregates each chromosome
independently with probability $p$.
Let $K \sim \mathrm{Binomial}(N,p)$ be the number of missegregating chromosomes.
Conditional on $K$, the net change in chromosome number for one daughter is
\[
\Delta = K - 2M,
\qquad
M \sim \mathrm{Binomial}(K,1/2).
\]
We truncate the distribution at symmetric limits $[-T,T]$, where
\[
T = \min\!\left(N,
\left\lceil z \sqrt{Np}\right\rceil\right),
\qquad
z = \Phi^{-1}\!\left(1 - \tfrac{\varepsilon_{\rm tail}}{2}\right).
\]
Let $d_N(\Delta)$ be this truncated (and renormalized) probability mass
function, optionally multiplied by a survival factor $(1-\ell)^{K}$ to capture
missegregation lethality $\ell$.

\subsection*{Construction of the missegregation matrices}

For each mother chromosome number $N$, missegregation produces two daughters
at chromosome numbers $N+\Delta$ and $N-\Delta$.  
The layer-specific missegregation matrices are:

\[
B_0(i,j)
=
\sum_{\Delta}
2\, d_{N_j}(\Delta)\,
\mathbf{1}\{N_i = N_j + \Delta\},
\qquad
B_1 \text{ defined analogously.}
\]

Out-of-range states are either discarded or clamped to the nearest boundary
(depending on the boundary rule).

\subsection*{Whole-genome doubling}

Pre–WGD states transition to post–WGD states with probability $p_{\rm WGD}$ per
division.  
A mother at $N_0$ produces two daughters at $2N_0$ with expected weight
$2p_{\rm WGD}$.
The corresponding block matrix $B_W$ satisfies
\[
B_W(i,j)
=
2\, p_{\rm WGD}(N_{0,j})\,
\mathbf{1}\{N_{1,i} = 2N_{0,j}\}.
\]

\subsection*{Full generator matrix}

Define diagonal matrices of division rates and WGD probabilities,
\[
L_0 = \mathrm{diag}(\lambda_0),\quad
L_1 = \mathrm{diag}(\lambda_1),\quad
S_0 = \mathrm{diag}(1-p_{\rm WGD}),\quad
S_W = \mathrm{diag}(p_{\rm WGD}).
\]

The generator $G$ has block form
\[
G =
\begin{pmatrix}
B_0 S_0 L_0 - L_0 & 0 \\[6pt]
B_W S_W L_0 & B_1 L_1 - L_1
\end{pmatrix}.
\]
This corresponds to the continuous-time limit of a multitype branching process,
where the $-L$ blocks encode the departure of mothers due to division and the
$B\,L$ blocks encode daughter production.

\subsection*{Discrete-time integration (simulation)}

We evolve the expected-state vector via an explicit Euler step
\[
x(t+\Delta t) \approx (I + \Delta t \, G)\, x(t).
\]
We use $\Delta t = 0.1$.
Population “passages” are defined by evolving until the total population size
has increased by a factor of~10; the time required for this defines the passage
duration.

\bigskip

\section*{Parameter estimation}

Model parameters are estimated by minimizing a joint cost consisting of:

\paragraph{(i) Ploidy-distribution negative log-likelihood.}
For each experiment and passage~$p$, we observe chromosome numbers
$N^{\text{obs}}$.
Given the predicted fractions $f_G(N)$ at the same passage,
the contribution to the negative log-likelihood is
\[
\mathrm{NLL}
=
-
\sum_N 
\mathrm{count}(N^{\text{obs}}{=}N)
\;
\log\bigl(f_G(N)+\varepsilon\bigr),
\]
with $\varepsilon=10^{-10}$ for numerical stability.

\paragraph{(ii) Growth-rate squared error.}
Observed passage durations yield empirical growth rates
$g_{\mathrm{obs}} = \log(10)/t_{\mathrm{obs}}$.
Model-predicted durations provide
$g_{\mathrm{sim}} = \log(10)/t_{\mathrm{sim}}$.
The cost is
\[
\mathrm{SSE}
=
\sum_{\text{passages}}
\bigl(g_{\mathrm{sim}} - g_{\mathrm{obs}}\bigr)^{2}.
\]

\paragraph{Total cost.}
\[
\mathrm{Cost}
=
\mathrm{NLL}
+
\mathrm{SSE}.
\]

\subsection*{Optimization procedure}

We optimize eight parameters:
\[
\theta = 
(R,\;\beta,\;\eta,\;p_{\rm WGD},\;\ell_0,\;\ell_1,\;p_{\rm mis}^{(O_2=1)},\;p_{\rm mis}^{(O_2=0)}).
\]
Parameters constrained to $(0,\infty)$ are optimized in log\(_{10}\) space.

We use the \texttt{DEoptim} genetic algorithm, running
100 generations in 10 successive batches.
After each batch, the current best parameters are used to regenerate diagnostic
plots comparing predicted vs.\ observed ploidy distributions and growth rates.
The final best parameters are stored in an RData file.

\bigskip

\section*{Reproducibility}

All simulations are performed in R using sparse matrices from the
\texttt{Matrix} package.

To reproduce the results:

\begin{enumerate}
\item Clone the repository and ensure the following files exist:
  \begin{itemize}
  \item \texttt{code/model\_functions.R} (all model definitions),
  \item \texttt{data/ploidy\_distribution.csv},
  \item \texttt{data/fit\_g.Rds}.
  \end{itemize}

\item Install required packages:
\begin{verbatim}
install.packages(c("Matrix","dplyr","tidyr","ggplot2","DEoptim","data.table"))
\end{verbatim}

\item Source the model and run the optimization:
\begin{verbatim}
source("path/to/the/main_script.R")
\end{verbatim}

\item Intermediate diagnostic plots are written to 
\texttt{plots/tmp/}.
The final optimized parameter set is saved as:
\[
\texttt{optim\_results.RData}.
\]

\item To rerun simulations at the optimized parameters, load that file and call:
\begin{verbatim}
sim <- run_all_sims(optim_params_real)
\end{verbatim}
\end{enumerate}

This provides full reproducibility of the parameter inference, the forward
simulations, and all manuscript figures derived from the model.


\end{document}
